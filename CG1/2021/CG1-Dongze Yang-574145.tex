\documentclass[fleqn]{article}
\usepackage[UTF8]{ctex}
\usepackage{listings}
\usepackage{pdfpages}
\usepackage{color}
\usepackage[colorlinks,linkcolor=blue]{hyperref}
\usepackage{dashrule}
\usepackage{diagbox}
\usepackage[german]{babel}
\usepackage[T1]{fontenc}
\usepackage[latin1]{inputenc}
\usepackage{titlesec}
\usepackage{geometry}
\usepackage{qtree}
\usepackage{tikz}
\usepackage{amsmath}
\usepackage{amssymb}
\setcounter{secnumdepth}{0}
\usetikzlibrary{positioning}
\geometry{top=2.5cm, bottom=2.5cm}
\lstset{
 columns=fixed,       
 numbers=left,                                        % 在左侧显示行号
 numberstyle=\tiny\color{gray},                       % 设定行号格式
 frame=none,                                          % 不显示背景边框
 backgroundcolor=\color[RGB]{245,245,244},            % 设定背景颜色
 keywordstyle=\color[RGB]{40,40,255},                 % 设定关键字颜色
 numberstyle=\footnotesize\color{darkgray},           
 commentstyle=\it\color[RGB]{0,96,96},                % 设置代码注释的格式
 stringstyle=\rmfamily\slshape\color[RGB]{128,0,0},   % 设置字符串格式
 showstringspaces=false,                              % 不显示字符串中的空格
 language=c++,                                        % 设置语言
 breaklines,                                          % 自动换行
}

%\title{CG1 Zusammenfassung WS20/21}

% \author{Dongze Yang}

\begin{document}

%\maketitle

%\tableofcontents

\newpagestyle{main}{
    \sethead{}{}{Dongze Yang 574145}
    \setfoot{}{\thepage}{}
    \headrule
    \footrule
}
\pagestyle{main}

\newpage

\section{Aufgabe 1}

\noindent\textbf{a) and b)}
\\
\\
\noindent Vertex Verarbeitung:

Vertex Shader:

$\circ$ Transformation Object Space $\rightarrow$ Clip Space

$\circ$ Transformation Positionen und Normalen für Beleuchtung

$\circ$ Gouraud Shading

\noindent Primitiv Verarbeitung:

$\circ$ Clipping im Clip Space

$\circ$ Transformation Clip Space $\rightarrow$ Windows space

$\circ$ Backface Culling

\noindent Rasterisierung

$\circ$ Punkt-/Linien-/Dreiecks-Rasterisierung im Windows Space

\noindent Fragment Verarbeitung:

$\circ$ Depth Test

$\circ$ Fragment Shader

$\circ$ Blending
\\
\\
\noindent\textbf{c)}

Rasterisierung

\section{Aufagbe 2}

\noindent\textbf{a)}

$1920\times 1280\times (32+16) = 117964800$ Bit

\noindent\textbf{b)}

Single-Buffering kann Bildschirmrissen kommen, 

Single-Buffering synchronisiert nicht mit dem Renderablauf $\Rightarrow$ Ausgabe von unfertigen Bildern während des Rendervorgangs.

Double-Buffering kann diese Problem lösen.

Doppelte Framebuffergröße gegenüber einfacher Pufferung, weil sie zwei Color Buffers verwendet:

\noindent\textbf{c)}

Die dreifache Pufferung behebt den Fehler "Wenn die Zeichengeschwindigkeit zu langsam ist, wird derselbe Rahmen mindestens zweimal auf dem Bildschirm angezeigt"

Der erforderliche Speicher ist dreimal so groß wie Single-Buffering.

\noindent\textbf{d)}

geringfügig (bis zur Dauer eines Frames) steigende und stärker fluktuierende Latenz

\noindent\textbf{e)}

$$Single-Bufferin$$

$$f_1+f_2+f_3+f_4+f_5=85ms$$

$$Double-Buffering\, ohne ,,VSync''$$

$$f_1+f_2+f_3+f_4+f_5=85ms$$

$$Double-Buffering\, mit ,,VSync''$$

$$\sum^5_{i=1}20ms\cdot \left\lceil\frac{20ms}{f_i}\right\rceil=160ms$$

\section{Aufgabe 3}

\noindent\textbf{a)}

$S_1$: Object Space

$S_2$: World Space

$S_3$: Eye Space

$S_4$: Clip Space

$S_5$: Normal Device Space

$S_6$: Window Space

M: Model Matrix

V: View Matrix

P: Projektions Matrix

A: Abbildung Matrix

1. Abbildung des Sichtvolumens auf achsparellelen Quader: M

2.  Position des Objektes auf dem Bildschirm: V

3.  Ausdehnung des Sichtvolumens entlang der Blickrichtung: P

4.  Ergebnis des Tiefenvergleichs beim Tiefentest: A

5.  Anzahl der Ausführungen des Vertex-Shaders im Draw Call: M V P

6.  Position der Kamera in der Welt: V
\\
\\
\noindent\textbf{b)}

$A_{-1}$ und $A'$

Verwenden Sie $A_{-1}$, um die vorhandenen Bildkoordinaten in NDC-Koordinaten zu ändern,
 und übergeben Sie dann ein neues $A$, damit es der neuen Auflösung entspricht.

$A=\begin{pmatrix}
    \frac{1}{2}\cdot 2400&0&0&\frac{1}{2}\cdot 2400+0\\
    0&\frac{1}{2}\cdot 1800&0& \frac{1}{2}\cdot 1800+0\\
    0&0&\frac{1}{2}\cdot(1-0)&\frac{1}{2}\\
    0&0&0&1
\end{pmatrix}\Rightarrow A^{-1}=\begin{pmatrix}
    \frac{1}{1200}&0&0&-1\\
    0&\frac{1}{900}&0&-1\\
    0&0&2&-1\\
    0&0&0&1
\end{pmatrix}$

$A'=\begin{pmatrix}
    \frac{1}{2}\cdot 5600&0&0&\frac{1}{2}\cdot 5600+0\\
    0&\frac{1}{2}\cdot 2400&0& \frac{1}{2}\cdot 2400+0\\
    0&0&\frac{1}{2}\cdot(1-0)&\frac{1}{2}\\
    0&0&0&1
\end{pmatrix}=\begin{pmatrix}
    2800&0&0&2800\\
    0&1200&0& 1200\\
    0&0&\frac{1}{2}&\frac{1}{2}\\
    0&0&0&1
\end{pmatrix}$

\noindent\textbf{c)}

1 mal M: Es gibt nur eine Szene, deshalb brauchen wir nur 1 Modell Matrix

4 mal V: Es gibt 4 Würfel unterschiedlicher Größe, d. h. haben wir 4 verschiedene Viewpoint.

4 mal P: verschiedene Viewpoint haben verschiedene Projection Matrix

1 mal A: Wenn wir nur 1 Anzeigegerät haben, brauchen wir nur 1 A Matrix.

\section{Aufgabe 4}

\noindent\textbf{a)}

Symmetrisierung, Skalier mit den Faktoren (1,2,1),  dreh um $90^\circ$, Transformation mit Vektor (-2,4,1)

$M_{Symm}=\begin{pmatrix}
    -1&0&0&0\\
    0&1&0&0\\
    0&0&1&0\\
    0&0&0&1
\end{pmatrix},\,
R=\begin{pmatrix}
    cos90^\circ&-sin90^\circ&0&0\\
    sin90^\circ&cos90^\circ&0&0\\
    0&0&1&0\\
    0&0&0&1
\end{pmatrix},\,
M_{Skl}=\begin{pmatrix}
    1&0&0&0\\
    0&2&0&0\\
    0&0&1&0\\
    0&0&0&1
\end{pmatrix},\,$ $
T=\begin{pmatrix}
    1&0&0&-2\\
    0&1&0&4\\
    0&0&1&0\\
    0&0&0&1
\end{pmatrix},\,
M=T\cdot R\cdot M_{Skl}\cdot M_{Symm}=\begin{pmatrix}
    0&-2&0&-2\\
    -1&0&0&4\\
    0&0&1&0\\
    0&0&0&1
\end{pmatrix},\,$

\noindent\textbf{b)}

$P=M^{-1}\cdot P=\begin{pmatrix}
    5\\0\\0\\1
\end{pmatrix}$

\section{Aufgabe 5}

\noindent\textbf{a)}

(b) $v_1$=(0,1),$v_2$=(0,0),$v_3$=(1,0),$v_4$=(1,1)

(c) $v_1$=(0.5,0.25),$v_2$=(1,0.25),$v_3$=(1,0.75),$v_4$=(0.5,0.75)

(d) $v_1$=(0,0),$v_2$=(2,0),$v_3$=(2,0.5),$v_4$=(0,2)

\noindent\textbf{c)}

\noindent Prinzip: 

statt einem Sample in der höchstmöglichen Stufelog $log_2\lceil \rho\rceil$ gew. Mittel von $n$ Samples in der niedrigeren Stufe $log_2\lceil \frac{\rho}{n}\rceil$.

Wahl vonnanhand des Verhältnisses längerer zu kürzerer Kante $\rightarrow$
hohe Bildqualität, sehr hoher Aufwand

\section{Aufgabe 6}

\noindent\textbf{b)}

\noindent left is Flat Shading

\noindent right isVertex Shader.$c=c_0\cdot\alpha+c_1\cdot\beta+c_2\cdot\gamma$

Auswertung der Beleuchtungsgleichung an allen Eckpunkten des Primitivs ,,per-Vertex Lighting''

\section{Aufgabe 8}

\noindent\textbf{c)}

a)es ist unmöglich, wenn y3 == y2 und y3 == y2 + 1 ist.


b)ähnlich wie (b). Es ist nicht möglich, wenn y5 == y4 und y5 == y4 + 1 ist.

c)Die Linie ist im Bresenham-Algorithmus immer steigend. D.h. gibt es keine sinkende Möglichkeit.

\section{Aufgabe 9}

Vorlesung folie: 07\_clipping (teil1).pdf Seite: 9













\end{document}