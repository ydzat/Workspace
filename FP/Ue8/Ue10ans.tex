\documentclass[fleqn]{article}
\usepackage[UTF8]{ctex}
\usepackage{listings}
\usepackage{pdfpages}
\usepackage{color}
\usepackage[colorlinks,linkcolor=blue]{hyperref}
\usepackage{dashrule}
\usepackage{diagbox}
\usepackage[german]{babel}
\usepackage[T1]{fontenc}
\usepackage[latin1]{inputenc}
\usepackage{titlesec}
\usepackage{geometry}
\usepackage{qtree}
\usepackage{tikz}
\usepackage{amsmath}
\usepackage{amssymb}
\setcounter{secnumdepth}{0}
\usetikzlibrary{positioning}
\geometry{top=2.5cm, bottom=2.5cm}
\lstset{
 columns=fixed,       
 numbers=left,                                        % 在左侧显示行号
 numberstyle=\tiny\color{gray},                       % 设定行号格式
 frame=none,                                          % 不显示背景边框
 backgroundcolor=\color[RGB]{245,245,244},            % 设定背景颜色
 keywordstyle=\color[RGB]{40,40,255},                 % 设定关键字颜色
 numberstyle=\footnotesize\color{darkgray},           
 commentstyle=\it\color[RGB]{0,96,96},                % 设置代码注释的格式
 stringstyle=\rmfamily\slshape\color[RGB]{128,0,0},   % 设置字符串格式
 showstringspaces=false,                              % 不显示字符串中的空格
 language=c++,                                        % 设置语言
 breaklines,                                          % 自动换行
}

% \title{TU Chemnitz}

% \author{Dongze Yang}

\begin{document}

\section{Ue 10 - IMP}

\noindent\textbf{Aufgabe 1}

\noindent \textbf{a)} $X:=(Y+1)*(Z-2)$

$$\frac{<Y,\sigma_0>\rightarrow0\,<1,\sigma_0>\rightarrow1}{<Y+1,\sigma_0>\rightarrow1}$$
$$\frac{<Z,\sigma_0>\rightarrow0\,<2,\sigma_0>\rightarrow2}{<Z-2,\sigma_0>\rightarrow-2}$$
$$\frac{<(Y+1)*(Z-2),\sigma_0>\rightarrow-2}{<X:=(Y+1)*(Z-2),\sigma_0>\rightarrow[-2/X]}$$
\\
\noindent \textbf{b)} 

\begin{center} (0)\qquad\textbf{if} $(2\leq 3) \wedge \,\neg$ \textbf{false then} $X:=1$ \textbf{else} $Y:=1$
\end{center}

$$(1)\qquad\frac{<2,\sigma_0>\rightarrow2\,<3,\sigma_0>\rightarrow3}{<2\leq 3,\sigma_0>\rightarrow true}$$
$$(2)\qquad\frac{<false,\sigma_0>\rightarrow false}{<\neg false,\sigma_0>\rightarrow true}$$
$$(3)\qquad\frac{(1)\qquad(2)}{<(2\leq 3)\wedge \neg false,\sigma_0>\rightarrow true}$$
$$(4)\qquad\frac{<1,\sigma_0>\rightarrow 1}{<X:=1,\sigma_0>\rightarrow[1/X]}$$
$$(5)\qquad\frac{<1,\sigma_0>\rightarrow 1}{<Y:=1,\sigma_0>\rightarrow[1/Y]}$$
$$(6)\qquad\frac{(3)\qquad(4)\qquad(5)}{<(0),\sigma_0>\rightarrow[1/X]}$$
\\
\noindent \textbf{c)} 

\begin{center}
    (0)\qquad    $X:=1;$ \textbf{while} $\neg (X = 2)$ \textbf{do} $X:=X+1$
\end{center}

$$(1)\qquad \frac{<1,\sigma_0>\rightarrow1}{<X:=1,\sigma_0>\rightarrow[1/X]}$$
$$(2)\qquad\frac{<X,\sigma_1>\rightarrow1\,<2,\sigma_1>\rightarrow2}{<(X=2),\sigma_1>\rightarrow false}$$
$$(3)\qquad\frac{(2)}{<\neg (X=2),\sigma_1>\rightarrow false}$$
$$(4)\qquad\frac{<X,\sigma_1>\rightarrow1\,<1,\sigma_1>\rightarrow1}{<X:=X+1,\sigma_1>\rightarrow[2/X]}$$
$$(5)\qquad\frac{(1)\qquad(3)\qquad(4)}{<while\neg(X=2)\,do\,X:=X+1,\sigma_1>\rightarrow\sigma_2}$$

$$(6)\qquad\frac{<X,\sigma_2>\rightarrow2\,<2,\sigma_2>\rightarrow2}{<(X=2),\sigma_2>\rightarrow true}$$
$$(7)\qquad\frac{(6)}{<while\neg(X=2)\,do\,X:=X+1,\sigma_2>\rightarrow\sigma_2}$$
\\
\noindent\textbf{Aufgabe 2}

\noindent\textbf{a)}

(I)\qquad\textbf{if} $b$ \textbf{then} $c_0$ \textbf{else} $c_1;c_2$

$$(1)\qquad\frac{\frac{\vdots}{<b,\sigma_0>\rightarrow true}\qquad\frac{\vdots}{<c_0,\sigma_0>\rightarrow\sigma_1}}{<(I),\sigma_0>\rightarrow\sigma_1}$$

$$(2)\qquad\frac{\frac{\vdots}{<b,\sigma_0>\rightarrow false}\qquad\frac{\vdots}{<c1;c2,\sigma_0>\rightarrow\sigma_2}}{<(I),\sigma_0>\rightarrow\sigma_2}$$




(II)\qquad\textbf{if} $b$ \textbf{then} $(c_0,c_2)$ \textbf{else} $(c_1;c_2)$

$$(1)\qquad\frac{\frac{\vdots}{<b,\sigma_0>\rightarrow true}\qquad\frac{\vdots}{<c_0;c_2,\sigma_0>\rightarrow\sigma_1}}{<(II),\sigma_0>\rightarrow\sigma_1}$$

$$(2)\qquad\frac{\frac{\vdots}{<b,\sigma_0>\rightarrow false}\qquad\frac{\vdots}{<c1;c2,\sigma_0>\rightarrow\sigma_2}}{<(II),\sigma_0>\rightarrow\sigma_2}$$

denn gilt: (I) (II) Äquivalenz ist. Weil (I)(1) und (II)(1) $\rightarrow\,\sigma_1$, (I)(2) und (II)(2) $\rightarrow\,\sigma_2$

\noindent\textbf{b)}

(III)\qquad\textbf{while} (n = 0) \textbf{do} (c; n ∶= 1) 

$$(1)\qquad\frac{\frac{\vdots}{<n=0,\sigma_0>\rightarrow true}\qquad\frac{\vdots}{<skip,\sigma_0>\rightarrow \sigma_0}}{<(III),\sigma_0>\rightarrow\sigma_0}$$

$$(2)\qquad\frac{\frac{\vdots}{<n=0,\sigma_0>\rightarrow false}\qquad\frac{\vdots}{<c;n=1,\sigma_0>\rightarrow\sigma_1}}{<(III),\sigma_0>\rightarrow\sigma_1}$$

(IV)\qquad\textbf{if} $¬(n = 0)$ \textbf{then} skip \textbf{else} (c; n ∶= 1)

$$(1)\qquad\frac{\frac{\vdots}{<\neg(n=0),\sigma_0>\rightarrow true}\qquad\frac{\vdots}{<skip,\sigma_0>\rightarrow \sigma_0}}{<(IV),\sigma_0>\rightarrow\sigma_0}$$

$$(2)\qquad\frac{\frac{\vdots}{<\neg(n=0),\sigma_0>\rightarrow false}\qquad\frac{\vdots}{<c;n=1,\sigma_0>\rightarrow \sigma_1}}{<(IV),\sigma_0>\rightarrow\sigma_1}$$

denn gilt: (III) (IV) Äquivalenz ist. Weil (III)(1) und (IV)(1) $\rightarrow\,\sigma_0$, (III)(2) und (IV)(2) $\rightarrow\,\sigma_1$

\end{document}