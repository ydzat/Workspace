\documentclass[fleqn]{article}
\usepackage[UTF8]{ctex}
\usepackage{listings}
\usepackage{diagbox}
\usepackage[german]{babel}
\usepackage[T1]{fontenc}
\usepackage[latin1]{inputenc}
\usepackage{titlesec}
\usepackage{geometry}
\usepackage{qtree}
\usepackage{tikz}
\usepackage{extarrows}
\usepackage{amsmath}
\usepackage{amssymb}
\setcounter{secnumdepth}{0}
\usetikzlibrary{positioning}
\geometry{top=2.5cm, bottom=2.5cm}
\lstset{
 columns=fixed,       
 numbers=left,                                        % 在左侧显示行号
 numberstyle=\tiny\color{gray},                       % 设定行号格式
 frame=none,                                          % 不显示背景边框
 backgroundcolor=\color[RGB]{245,245,244},            % 设定背景颜色
 keywordstyle=\color[RGB]{40,40,255},                 % 设定关键字颜色
 numberstyle=\footnotesize\color{darkgray},           
 commentstyle=\it\color[RGB]{0,96,96},                % 设置代码注释的格式
 stringstyle=\rmfamily\slshape\color[RGB]{128,0,0},   % 设置字符串格式
 showstringspaces=false,                              % 不显示字符串中的空格
 language=c++,                                        % 设置语言
 breaklines,                                          % 自动换行
}

%\title{TU Chemnitz \\ Praktikum Grundlagen Technische Informatik \\ Versuch Sequ1}

%\author{Gruppe 5 - Team 5: \\ Dongze Yang \\Xiangyu Tong \\ Treshchun Kateryna}

\begin{document}

%\maketitle

\newpagestyle{main}{
    \sethead{}{}{Math.Geo}
    \setfoot{}{\thepage}{}
    \headrule
    \footrule
}
\pagestyle{main}

\section{K 1}

\noindent 1. $R^3$中两点距离(可推导至$R^n$):$|AB|=\sqrt{(x_1^2-x_2^2)^2+(y_1^2-y_2^2)^2+(z_1^2-z_2^2)}$

\noindent 2. 已知3点求三角形面积:$S=\sqrt{P(P-A)(P-B)(P-C)}$,其中$P=(A+B+C)/2$,A/B/C为边长。边长用两点距离公式算。或:$S=\frac{1}{2}|\overrightarrow{AB}\times\overrightarrow{AC}|$。

\noindent 3. 数量积/内积/点积 Skalatprodukt: $\vec{a}\circ\vec{b}=a_1b_1+a_2b_2+\dots+a_nb_n$。是标量。

几何定义:$\vec{a}\cdot\vec{b}=|\vec{a}|\cdot|\vec{b}|\cdot \cos\theta,\,0\leq \theta\leq \pi$
\\
\\
\noindent 4. 叉积/向量积 Kreuzprodukt

$\vec{a}\times\vec{b}=\begin{pmatrix}
    a_1 \\ a_2\\a_3
\end{pmatrix}\times\begin{pmatrix}
    b_1 \\ b_2\\b_3
\end{pmatrix}=\begin{pmatrix}
    v_1 \\ v_2\\v_3
\end{pmatrix}=\begin{pmatrix}
    a_2b_3-a_3b_2 \\ a_3b_1-a_1b_3\\a_1b_2-a_2b_1
\end{pmatrix}$
\\
\\
\noindent 5.线性相关/无关 linear (un)abhängig (不)在一条线上或平行。

Bsp. 有$\vec{a}=(x_1,y_1,z_1),\vec{b}=(x_2,y_2,z_2)\vec{c}=(x_3,y_3,z_3)$。

则:$\begin{pmatrix}
    \vec{a}\\\vec{b}\\\vec{c}
\end{pmatrix}=\begin{pmatrix}
    x_1&y_1&z_1\\x_2&y_2&z_2\\x_3&y_3&z_3
\end{pmatrix}$, 通过线性变换(行列相加减)$\rightarrow\begin{pmatrix}
    x&x&x\\x&x&x\\0&0&0
\end{pmatrix}$则线性相关,意味着任意向量都可以由另外两个向量表示。
\\
\\
\noindent 6. 三重积/混合积 Spatprodukt $\rightarrow$三个向量相乘

$\vec{a}\cdot(\vec{b}\times\vec{c}) = \begin{bmatrix}
    a_1&a_2&a_3\\b_1&b_2&b_3\\c_1&c_2&c_3
\end{bmatrix}=-\vec{b}\cdot(\vec{a}\times\vec{c})=-\vec{c}\cdot(\vec{a}\times\vec{b})=[a\,b\,c]$。若任意两个向量相等,则=0。

由此推导出体积公式$V_{Spat}=|\vec{a}\cdot(\vec{b}\times\vec{c})|=\left| det\begin{pmatrix}
    a_1&a_2&a_3\\b_1&b_2&b_3\\c_1&c_2&c_3
\end{pmatrix}\right|$,四面体体积$V_{Tetraeders}=\frac{1}{6}V$

\noindent 7. 已知$\vec{a}=(x_1,y_1,z_1),\vec{b}=(x_2,y_2,z_2)$,判断两个向量的空间关系:

$\vec{a}\parallel\vec{b}\Rightarrow c=\frac{x_2}{x_1}=\frac{y_2}{y_1}=\frac{z_2}{z_1}$

$\vec{a}\perp\vec{b}\Rightarrow \vec{a}\cdot\vec{b}=x_1x_2+y_1y_2+z_1z_2=0$

Windschief异面: 既不相交也不平行。

\noindent 8. 2个向量函数距离:

首先,判断两个函数是否$\parallel$,若是,则:在函数上取变量值相同的两点$P_1,P_2,P\in x_1(t),P_2\in x_2(s)$,求$\overrightarrow{P_1P_2}$。
将$x_1(t)$求导,取得其导函数,其为这个原函数的切向量,即方向向量$\vec{r}$。随后可求$d(g_1,g_2)=d(P_2,g_1)=\frac{|\overrightarrow{P_1P_2}\times\vec{r}|}{|\vec{r}|}$。
其中$|\overrightarrow{P_1P_2}\times\vec{r}|$表示$\vec{r}$与$\overrightarrow{P_1P_2}$组成的平行四边形面积。

若两向量异面,则:求$g_1,g_2$的垂直向量/法向量Lotvektor $\vec{n}=\vec{r_1}\times\vec{r_2}$。在函数上取变量值相同的两点$P_1,P_2,P\in x_1(t),P_2\in x_2(s)$,求$\overrightarrow{P_1P_2}$。
则有$d(g_1,g_2)=\frac{|\overrightarrow{P_1P_2}\times\vec{n}|}{|\vec{n}|}$。
\\
\\
\noindent 9. 点到向量函数距离 

在函数上取一点S,求$\overrightarrow{SP}$,则$d(P,g)=\frac{|\overrightarrow{SP}\times\vec{r}|}{|\vec{r}|}$

\noindent 10. 三元一次方程组$ax+by+cz=d$的法向量Normalenvektor $\vec{n} = \begin{pmatrix}
    a\\b\\c
\end{pmatrix}$

\noindent 11. 过指定点的平面E与向量函数$g: \begin{pmatrix}
    x(t)\\y(t)\\z(t)
\end{pmatrix}=t\cdot\begin{pmatrix}
    a\\b\\c
\end{pmatrix}$垂直,求E。

$\because E\parallel g \therefore E: ax+by+cz=d$。将g的数值代入E即可求得d。

其中$\begin{pmatrix}
    a\\b\\c
\end{pmatrix}$为g的方向向量。

\noindent 12. 向量函数与E的交点:

将$g: \begin{pmatrix}
    x(t)\\y(t)\\z(t)
\end{pmatrix}$带入平面方程函数,即可求得t,将t带入g即可。

\noindent 13. 求两平面相交线的线性方程(用矩阵)

已知$E_1: -2x+4y-2z=-6,\,E_2: x+7y+z=21$

解:$\begin{matrix}
    a:\\b:
\end{matrix}\begin{vmatrix}
    -2&4&-2&-6\\1&7&1&21
\end{vmatrix}(1)\xlongequal[]{a-(-2)b}=\begin{vmatrix}
    -2&4&-2&-6\\0&18&0&36
\end{vmatrix}\Rightarrow y=2$

$(1)\xlongequal[]{7a-4b}=\begin{vmatrix}
    -14&28&-14&-42\\4&28&4&84
\end{vmatrix}\xlongequal[]{b=a-b}=\begin{vmatrix}
    -14&28&-14&-42\\-18&0&-18&-126
\end{vmatrix}\Rightarrow z=7-x$

设$x=0$,则$z=7$,可知向量方程一定过点(0,2,7),即$\vec{S}=\begin{pmatrix}
    0\\2\\7
\end{pmatrix}$。

直线方向向量 = 两平面法向量叉积$\vec{r}=\vec{n_1}\times\vec{n_2}$,可得$\vec{r}$,即$g: x(t)=\vec{S}+t\cdot\vec{r}$。

\noindent 14. 平面向量方程$\rightarrow$一般平面方程

已知:$\begin{pmatrix}
    x(s,t)\\y(s,t)\\z(s,t)
\end{pmatrix}=\begin{pmatrix}
    1\\3\\-3
\end{pmatrix}+t\cdot\begin{pmatrix}
    1\\-4\\0
\end{pmatrix}+s\cdot\begin{pmatrix}
    -5\\0\\3
\end{pmatrix}$

则:$\begin{pmatrix}
    1\\-4\\0
\end{pmatrix}\times\begin{pmatrix}
    -5\\0\\3
\end{pmatrix}=\begin{pmatrix}
    a\\b\\c
\end{pmatrix}\Rightarrow ax+by+cz=d$,代入$\begin{pmatrix}
    1\\3\\-3
\end{pmatrix}$即可。

\section{K 2}

\noindent 1. Homogenen Koordinaten 齐次坐标/投影坐标(Projktion Koordinaten)

功能:可以让无穷远的坐标以有限坐标表示。

使用齐次坐标可以进行仿射变换(线性变换+平移),其投影变换可以用矩阵表示。

\noindent 2. 给定Euklidische Geometrie(欧几里得平面)上一点(x,y),对于任意非零实数z,三元组(xz,yz,z)即为该点的齐次坐标。

\noindent 3. 一条过原点的线: nx+my=0,其中n和m不同时为0,以参数表示可写成$\left\{\begin{aligned}
    x=mt\\y=-nt
\end{aligned}\right.$,另$z=1$,则线上的Kartesischen Koordinaten(笛卡尔坐标)为$(m/z,-n/z)$,在齐次坐标下为$(m,-n,z)$。当$t\rightarrow\infty$时,齐次坐标会变成$(m,-n,0)$。
注:原点为$(0,0,1)$
        
\noindent 4. 在n维中的投影可表示为n+1元组,例如在$R^3$中的点$P(x,y,z)$的齐次坐标为$(wx,wy,wz,w)$。
\\
\\
\noindent 5. 仿射变换 Affine Abbildung

$R^2:\vec{y}=A\vec{x}+\vec{b}\Rightarrow\begin{bmatrix}
    \vec{y}\\1
\end{bmatrix}=\begin{bmatrix}
    A&\vec{b}\\0&1
\end{bmatrix}\cdot\begin{bmatrix}
    \vec{x}\\1
\end{bmatrix}$

假设$\vec{x}$是$R^2$上正方形任意一点的坐标,则$\begin{bmatrix}
    \vec{x}\\1
\end{bmatrix}$表示把正方形投影到了$R^3$中$z=1$的平面上。

即:增加一个维度后,可以再高维通过线性变换完成低维度的仿射变换。

\noindent 6. 通过齐次坐标,可以用矩阵乘法来描述位移或平移。

如:已知$M=\begin{pmatrix}
    \tilde{x}\\\tilde{y}\\\tilde{z}
\end{pmatrix}:=\begin{pmatrix}
    x+2\\y+7\\z+1
\end{pmatrix}$,求其平移矩阵A。

则有$\begin{pmatrix}
    \tilde{x}\\\tilde{y}\\\tilde{z}\\1
\end{pmatrix}=A\begin{pmatrix}
    x\\y\\z\\1
\end{pmatrix}\Rightarrow A=\begin{pmatrix}
    1&0&0&2\\
    0&1&0&7\\
    0&0&1&1\\
    0&0&0&1
\end{pmatrix}$

即平移矩阵为$\begin{pmatrix}
    I&Verschiebung\\
    0&1
\end{pmatrix}$
\\
\\
\noindent 7. Baryzentrischen Kooridinaten 重心坐标

若$A,B,C,D \in R^3$,则$P_{ABCD}=(\alpha,\beta,\theta,\gamma)$,求P的笛卡尔坐标。($\alpha+\beta+\theta+\gamma=1$)

则:$\alpha\cdot A+\beta\cdot B + \theta\cdot C + \gamma\cdot D = P$,其中ABCD的坐标以向量形式表示。

判断重心位置:(1)若$\alpha,\beta,\theta,\gamma\in(0,1)$,则在内。(2)若$\notin (0,1)$,则需要根据其越域参数来判断:如$\alpha<0$,则重心位于bcd平面之下,若$\alpha>0$,则位于bcd平面的上方。若为0,则置于平面中。

\noindent 8. 求Transformationsmatrix。

已知Quader长方体:$\{\begin{pmatrix}
    x\\y\\z
\end{pmatrix}:2\leq x\leq6;1\leq y\leq 2;10\leq z\leq 40\}$,要把它投影成Einheitswürfel单位方块$\{\begin{pmatrix}
    x_i\\y_i\\z_i
\end{pmatrix}:0\leq x\leq 1, 1\leq y\leq 1, 0\leq z \leq 1\}$

解:先平移再缩放。

$x_i \rightarrow \frac{x-2}{6-2} = \frac{1}{4}x-\frac{1}{2}$

$y_i \rightarrow \frac{y-1}{2-1} = y-1$

$z_u \rightarrow \frac{z-10}{40-10}=\frac{1}{30}z-\frac{1}{3}$

$\begin{pmatrix}
    x_i\\y_i\\z_i\\1
\end{pmatrix}=T\cdot\begin{pmatrix}
    x\\y\\z\\1
\end{pmatrix}\Rightarrow T=\begin{pmatrix}
    skaliert M&Verschiebung\\
    0&1
\end{pmatrix}=\begin{pmatrix}
    \frac{1}{4}&0&0&-\frac{1}{2}\\
    0&1&0&-1\\
    0&0&\frac{1}{30}&-\frac{1}{3}
    0&0&0&1
\end{pmatrix}$
\\
\\
\noindent 9. Skalierung 缩放:把一个向量/点/方向,沿各自方向放大/缩小。$T_{R^3}=\begin{bmatrix}
    x&0&0&0\\
    0&y&0&0\\
    0&0&z&0\\
    0&0&0&1
\end{bmatrix}$

\noindent 10. Rotationsmatrix 旋转矩阵

$R^2$,
关于原点逆时针旋转$\theta$°:$\begin{pmatrix}
    \cos\theta&-\sin\theta\\
    \sin\theta&\cos\theta
\end{pmatrix}$

$R^3$:右手坐标系,大拇指指向某轴正方向,四指弯曲方向为正。

(1)关于$x$轴:$\begin{pmatrix}
    1&0&0\\
    0&\cos\theta&-\sin\theta\\
    0&\sin\theta&\cos\theta
\end{pmatrix}$;
(2)y:$\begin{pmatrix}
    \cos\theta&0&-\sin\theta\\
    0&1&0\\
    \sin\theta&0&\cos\theta
\end{pmatrix}$;
(3)z:$\begin{pmatrix}
    \cos\theta&-\sin\theta&0\\
    \sin\theta&\cos\theta&0\\
    0&0&1
\end{pmatrix}$
\\
\\
\noindent \textbf{累积变换(Kumulative Transformation): 当需要执行不同动作时:先缩放,中旋转,后平移。}
\\
\\
\noindent 11. 求T:(1)先平移,使点$(4,9,2)^T$成为新的原点。(2)然后将其沿着x,y,z方向以5,3,2为单位延伸(streckt)。(3)最后沿z轴正方向旋转30°。

$\left\{\begin{aligned}
    x'&=5(x-4)=5x-20\\
    y'&=3(y-9)=3y-27\\
    z'&=2(z-2)=2z-4
\end{aligned}\right.\quad\Rightarrow T'=\begin{pmatrix}
    5&0&0&-20\\
    0&3&0&-27\\
    0&0&2&-4\\
    0&0&0&1
\end{pmatrix}\quad\Rightarrow\begin{pmatrix}
    x_{new}\\y_{new}\\z_{new}
\end{pmatrix}=T_{z30°}\cdot T' \cdot \begin{pmatrix}
    x\\y\\z
\end{pmatrix}$

$\therefore T=T_{30°}\cdot T'=\begin{pmatrix}
    \cos30°&-\sin30°&0&0\\
    \sin30°&\cos30°&0&0\\
    0&0&1&0\\
    0&0&0&1
\end{pmatrix}\cdot\begin{pmatrix}
    5&0&0&-20\\
    0&3&0&-27\\
    0&0&2&-4\\
    0&0&0&1
\end{pmatrix}=\begin{pmatrix}
    \frac{5\sqrt{3}}{2}&-\frac{3}{2}&0&\frac{27-20\sqrt{3}}{2}\\
    \frac{5}{2}&\frac{3\sqrt{3}}{2}&0&\frac{-27\sqrt{3}-20}{2}\\
    0&0&2&-4\\
    0&0&0&1
\end{pmatrix}$

\noindent 12. 特殊矩阵

关于x轴的镜像$\begin{pmatrix}
    1&0\\0&-1
\end{pmatrix}$;关于y轴的镜像$\begin{pmatrix}
    -1&0\\0&1
\end{pmatrix}$;关于y=x轴的镜像/翻转$\begin{pmatrix}
    0&1\\1&0
\end{pmatrix}$;
缩小至$\frac{1}{2}:\begin{pmatrix}
    \frac{1}{2}&0\\0&\frac{1}{2}
\end{pmatrix}$;

在y=x上的投影$\begin{pmatrix}
    \frac{1}{2}&\frac{1}{2}\\\frac{1}{2}&\frac{1}{2}
\end{pmatrix}$;剪切,每个坐标向每个其他坐标移动1倍$\begin{pmatrix}
    1&1&1&0\\
    1&1&1&0\\
    1&1&1&0\\
    0&0&0&1
\end{pmatrix}$;

$R^2$关于某线的镜像矩阵:$\begin{pmatrix}
    1-2x^2&-2xy\\-2xy&1-2y^2
\end{pmatrix}$;$R^3$关于某面的镜像:$\begin{pmatrix}
    1-2x^2&-2xy&-2xz\\
    -2xy&1-2y^2&-2yz\\
    -2xz&-2yz&1-2z^2
\end{pmatrix}$;

向任意过原点的直线或平面投影:$R^2:\begin{pmatrix}
    1-n_x^2&-n_xn_y\\-n_x&1-n_y^2
\end{pmatrix};R^3:\begin{pmatrix}
    1-2n_x^2&-2n_xn_y&-2n_xn_z\\
    -2n_xn_y&1-2n_y^2&-2n_yn_z\\
    -2n_xn_z&-2n_yn_z&1-2n_z^2
\end{pmatrix}$

上述的$x,y,z$或$n_x,n_y,n_z$皆为法向量/方向向量的单位向量,例如$\vec{n}=(5,-3,3)^T$,则$n_x=\frac{5}{\sqrt{5^2+(-3)^2+3^2}}$。

即$Spieglungsmatrix:I-2\vec{n}\vec{n}^T$.

\noindent 13. 正交 Orthogonal

旋转矩阵R是正交的:$<R_1,R_2>=0$,即任意两行的点积为0。任一行的平方=1。(每一行都是基向量)。$R\cdot R^T = I$。

通过原点的平面处镜像反射矩阵也是正交和对称的。这样的反射矩阵$S=I-2\cdot\vec{n}\cdot\vec{n}^T$,其中$\vec{n}$为镜面的法向量。

\noindent 14. 已知Parallelogramm(平行四边形)四点$\widetilde{A}(1,2),\widetilde{B}(5,3),\widetilde{C}(6,11),\widetilde{D}(2,10)$,
将其投影到单位正方形上,求转换矩阵T。

任选3点变成齐次坐标,则有$T\cdot A = \begin{pmatrix}
    0\\0\\1
\end{pmatrix}, T\cdot B = \begin{pmatrix}
    1\\0\\1
\end{pmatrix},T\cdot C=\begin{pmatrix}
    1\\1\\1
\end{pmatrix}\Rightarrow T\cdot \begin{pmatrix}
    A&B&C
\end{pmatrix}=\begin{pmatrix}
    0&1&1\\
    0&0&1\\
    1&1&1
\end{pmatrix}$,其中有$M=\begin{pmatrix}
    A&B&C
\end{pmatrix}=\begin{pmatrix}
    1&5&6\\
    2&3&11\\
    1&1&1
\end{pmatrix}\Rightarrow T\cdot M\cdot M^{-1}=T=\begin{pmatrix}
    0&1&1\\
    0&0&1\\
    1&1&1
\end{pmatrix}\cdot M^{-1}$。求出$M^{-1}$带入即可。
\\
\\
\noindent 15. 有Zylinder(圆柱体),两点$P_1,P_2$在它的轴(Achse)上,并且半径r=1。有平面E,其法向量为$\vec{n}$,该平面过点$P_0$。
E切开了Zylinder,通过T,切面变成了新的x-y平面的单位圆,平面得到法向量$\vec{n}$作为新的z轴$e_3=(0,0,1)^T$。

这个转换应该是可逆的(kehrbar),且无投影(keine Projektion)。

已知:$P_1=\begin{pmatrix}
    0\\0\\0
\end{pmatrix},P_2=\begin{pmatrix}
    0\\0\\1
\end{pmatrix},P_0=\begin{pmatrix}
    -1\\-1\\3
\end{pmatrix},\vec{n}=\frac{1}{5}\begin{pmatrix}
    -3\\0\\4
\end{pmatrix},E:\vec{x}=\begin{pmatrix}
    0\\0\\\frac{15}{4}
\end{pmatrix}+\begin{pmatrix}
    1\\0\\\frac{3}{4}
\end{pmatrix}\cdot t+\begin{pmatrix}
    0\\1\\0
\end{pmatrix}\cdot s$

则:




















































































\end{document}