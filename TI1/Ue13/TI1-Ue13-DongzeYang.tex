\documentclass[fleqn]{article}

\usepackage{listings}
\usepackage[german]{babel}
\usepackage[T1]{fontenc}
\usepackage[latin1]{inputenc}
\usepackage{titlesec}
\usepackage{geometry}
\usepackage{qtree}
\usepackage{tikz}
\usepackage{amsmath}
\setcounter{secnumdepth}{0}
\usetikzlibrary{positioning}
\geometry{top=2.5cm, bottom=2.5cm}
\lstset{
 columns=fixed,       
 numbers=left,                                        % 在左侧显示行号
 numberstyle=\tiny\color{gray},                       % 设定行号格式
 frame=none,                                          % 不显示背景边框
 backgroundcolor=\color[RGB]{245,245,244},            % 设定背景颜色
 keywordstyle=\color[RGB]{40,40,255},                 % 设定关键字颜色
 numberstyle=\footnotesize\color{darkgray},           
 commentstyle=\it\color[RGB]{0,96,96},                % 设置代码注释的格式
 stringstyle=\rmfamily\slshape\color[RGB]{128,0,0},   % 设置字符串格式
 showstringspaces=false,                              % 不显示字符串中的空格
 language=c++,                                        % 设置语言
 breaklines,                                          % 自动换行
}
\begin{document}

\newpagestyle{main}{
    \sethead{Matrikel-Nr.: 574145 Dongze Yang}{}{Grupe: Mi. Dozent: Julian Pape-Lange}
    \setfoot{}{\thepage}{}
    \headrule
    \footrule
}
\pagestyle{main}
%\section{Aufgabe}

\section{3. Aufgabe}

Schulmethode:
\\

\begin{tabular}{lr}
    &1100\\
    *&1011\\
    \hline\\
    &1100\\
    &1100 \ \! \\
    &0000 \ \! \ \! \\
    +&1100 \ \! \ \! \ \! \\
    \hline\\
    &10000100
\end{tabular}
\\
\\
Karazuba:

\begin{equation}
    \left\{
        \begin{array}{lr}
            1100 = x = x_h\cdot 2^{\frac{n}{2}} + x_l = 11\cdot 2^2 + 00\\
            1011 = y = y_h\cdot 2^{\frac{n}{2}} +y_l = 10\cdot 2^2 +11\\
            n = 4 \\
            P = (x_h + x_l)(y_h + y_l)
        \end{array}
    \right.
\end{equation}

$\Rightarrow x \cdot y = x_hy_h2^n+(P-x_hy_h-x_ly_l)2^{\frac{n}{2}} +x_ly_l$


\begin{equation}
    \begin{split}
        1100\times 1011 & = 11\times 10\times 2^4 + ((11+00)(10+11)-11\times 10 - 00\times 11)2^2+00\times 11 \\
        & = (1 \times 1 \times 2^2 + (10\times 1 - 1\times 1 - 1\times 0)2^1 + 0\times 1)2^4+(11\times 101 - 11\times 10)2^2 \\
        & = (100+(10-1)2^1)2^4 +(11\times 101 - 110)2^2 \\ 
        & = 110\times 2^4 + (1\times 10\times 2^2 + ((1+1)(10+1)-1\times 10 - 1\times 1)2^1 + 1\times 1 - 110)2^2\\
        & = 110\times 2^4 +1001\times 2^2\\
        & = 1000 \ 0100
    \end{split}
\end{equation}

\section{5. Aufgabe}

$F = (\neg A \vee B \vee C) \wedge (A \vee B \vee C) \wedge (A \vee B \vee \neg C)$

A = 0
\begin{equation}
\begin{split}
    & \ \ \ \ (1 \vee B \vee C) \wedge (0 \vee B \vee C) \wedge (0\vee B \vee \neg C)\\
    &= (B\vee C)\wedge (B\vee \neg C)
\end{split}
\end{equation}

B = 0

$\ \ \ \  = C \wedge \neg C$

B = 1

$ \ \ \ \ = (1\vee C)\wedge (1\vee \neg C)$

$ \ \ \ \ = 1$

return erfüllbar

\end{document}