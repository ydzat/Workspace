\newpage

\noindent \textbf{代码:3 无向图}

\begin{lstlisting}[language = java]
//uGraph.java
public class uGraph{

    private int[] v;            //默认节点从1开始,顺序递增,即1,2,...
    private int[][] e;          //初始化为空,若存在节点,则变为1。储存格式为二维数组,横纵角标对应节点编号,0为1,1为2...
    private int group;
    private Node[] nodes;       //储存所有节点,默认index = value - 1

    //双模式,flag = 0则是无向图,flag = 1则是有向图
    uGraph(int[] vt, int[][] et, int flag){
        this.v = vt;
        this.e = new int[v.length][v.length];
        for(int[] i : et){
            this.e[i[0]-1][i[1]-1] = 1;
            this.e[i[1]-1][i[0]-1] = 1;     //对于无向图来说,需要此行,若是有向图,则需要删除此行。
        }

        //对于无向图
        if(flag == 0){
            this.nodes = new Node[vt.length];
            for(int i = 0; i < vt.length; i++){
                //得到一行的grad
                int grad = 0;
                for(int j = 0 ; j < vt.length; j++){
                    grad = grad + e[i][j];
                }
                //录入节点
                //int value, int eGrad, int aGrad, int grad
                this.nodes[i] = new Node(this.v[i],0,0,grad);
            }
        }
        //对于有向图
        if(flag == 1){
            //to do
        }
    }

    public void getGraph(){
        System.out.print("  ");
        for(int i = 1; i <= v.length; i++){
            System.out.print(i + " ");
        }
        System.out.print("\n");
        for(int i = 0; i < v.length; i++){
            System.out.print(i+1 + " ");
            for(int j = 0; j < v.length; j++){
                System.out.print(this.e[i][j] + " " );
            }
            System.out.print("\n" );
        }
    }

    public int[] getV(){
        return this.v;
    }

    public int[][] getE(){
        return this.e;
    }

    public Node[] getNodes(){
        return this.nodes;
    }
}
\end{lstlisting}

\begin{lstlisting}[language = java]
//Farben.java
import java.util.LinkedList;
import java.util.Queue;
import java.util.Stack;

public class Farben{

    uGraph ugg;

    Farben(uGraph ug){
        this.ugg = ug;
    }

    public void getUgg(){
        this.ugg.getGraph();
    }

    //ob 3-färbbar sein
    public void fb3(){

        int lengthV = this.ugg.getV().length;
        //System.out.print("length = " + lengthV);
        int[] deleteList = new int[lengthV];
        int last = 0;       //last每次指向最新为空的下标
        int top = 0;        //top每次指向当前需要操作的序列的开头元素
        int flag = 0;       //若图中存在任一一个节点是小于等于2,则为1,否则为0

        while(true){
            //检索第一次,找出当前图中所有度数小于2的节点,记录在deleteList中
            //若当前图中所有节点的度都大于2,则退出循环
            //若last == lengthV,则退出循环
            if(last == lengthV){
                break;
            }
            for(int i = 0; i < lengthV; i++){
                int sum = 0;
                for(int j : this.ugg.getE()[i]){
                    sum = sum + j;
                }
                if (sum <= 2 && sum != 0){
                    deleteList[last] = i;
                    last ++;
                    flag = 1;
                }
            }
            if(flag == 0){
                break;
            }

            //第一轮删除,删除deleteList中指向的节点与边。即,置为0
            for(int i = top; i < last; i++){
                for(int j = 0; j < lengthV; j++){
                    this.ugg.getE()[deleteList[i]][j] = 0;
                    this.ugg.getE()[j][deleteList[i]] = 0;      //有向图中删去此行
                }
            }
            int t = top;
            top = last;

            //输出1次当前的G'
            System.out.print("///////////////////////////////////\n");
            //此时的deleteList
            System.out.print("deleteList is: ");
            for(int i = t; i < last; i++){
                System.out.print(deleteList[i]+1 + " ");
            }
            System.out.print("\n");
            this.ugg.getGraph();
            System.out.print("///////////////////////////////////\n");
        }

        //判断当前的G'中是否还有剩余,实际上判断last的值是否与lengthV相等即可
        if(last == lengthV){
            System.out.print("G ist 3-färbbar\n\n");
        }
        else{
            System.out.print("weiß nicht\n\n");
        }
    }

    //ob 2-färbbar / bipartit  sein, 二分图
    //思路:BFS或DFS,此处用BFS
    public void bipartit(){

        //要从每个结点为起始点,均进行一次BFS,因此一共需进行this.ugg.getV().length次。
        int lengthV = this.ugg.getV().length;

        for(int i = 0; i < lengthV; i++){
            //i+1的值为节点的实际值value
            System.out.print("von " + (i+1) + " starten:\n");

            //将.getE()更改行的位置,让第i行变为第0行,依次变换。
            int[][] eTemp = new int[lengthV][lengthV];
            for(int j = 0; j < lengthV; j++){
                eTemp[j] = this.ugg.getE()[(i+j)%lengthV];
            }

            //BFS
            //建立一个等待队列
            Queue<Node> warteSchlange = new LinkedList<Node>();
            //建立一个已遍历栈
            Stack<Node> expansion = new Stack<Node>();

            //BFS准备:先将当前初始节点入队
            warteSchlange.offer(ugg.getNodes()[i]);       //默认index = value - 1
            System.out.print("初始节点为:"+ ugg.getNodes()[i].getValue() + "\n");
            ugg.getNodes()[i].setGroup(0);
            int lauf = 0;

            while(true){
                //判断队列是否为空,若为空,则退出循环
                if(warteSchlange.isEmpty()){
                    break;
                }

                //将第一个元素出队,并放进已遍历栈中,并判断其相邻的节点
                //若本身group=0,则相邻节点的group = 1;否则=0。
                expansion.push(warteSchlange.poll());
                int groupTemp = expansion.peek().getGroup();

                for(int j = 0; j < lengthV; j++ ){
                    if(eTemp[lauf][j] == 1 && expansion.search(ugg.getNodes()[j]) == -1){
                        warteSchlange.offer(ugg.getNodes()[j]);     //若有边,则入队
                        if(groupTemp == 0){
                            ugg.getNodes()[j].setGroup(1);
                        }
                        else{
                            ugg.getNodes()[j].setGroup(0);
                        }
                    }
                }
                lauf++;
            }

            //输出遍历顺序:
            System.out.print("遍历顺序为:");
            for(Node j : expansion){
                System.out.print( j.getValue() + " ");
            }
            System.out.print("\n");

            //从i开始的一次遍历完成后,判断是否为二分图
            //即遍历所有边,看是否存在边的两端节点同属一个group,若是,则“不是二分图”
            //否则“是二分图”
            int flag = 0;
            for(int j = 0; j < lengthV; j++){
                for(int k = j+1; k < lengthV; k++){
                    if(this.ugg.getE()[j][k] == 1){
                        if(this.ugg.getNodes()[j].getGroup() == this.ugg.getNodes()[k].getGroup() ){
                            System.out.print("nicht 2-färbbar / bipartit\n");
                            flag = 1;
                            break;
                        }
                    }
                }
                if(flag == 1){
                    break;
                }
            }

            if(flag == 0){
                System.out.print("G ist 2-färbbar / bipartit !\n");

                //输出分组:
                Queue<Node> nodesGroup0 = new LinkedList<Node>();       //该组存储组0的节点
                Queue<Node> nodesGroup1 = new LinkedList<Node>();       //该组存储组1的节点
                
                //先分组
                for(Node j : this.ugg.getNodes()){
                    if(j.getGroup() == 0){
                        nodesGroup0.offer(j);
                    }
                    else{
                        nodesGroup1.offer(j);
                    }
                }

                //然后输出
                System.out.print("Group 0 hat:\n  ");
                for(Node j : nodesGroup0){
                    System.out.print(j.getValue() + " ");
                }
                System.out.print("\nGroup 1 hat:\n  ");
                for(Node j : nodesGroup1){
                    System.out.print(j.getValue() + " ");
                }
                System.out.print("\n");
            }
        }
    }
}
\end{lstlisting}

\begin{lstlisting}[language = java]
//Test.java
public class Test{
    public static void main(String[] args){
        //3-färbbar (a)
        int[] v1 = {1,2,3,4,5,6,7};
        int[][] e1 = {{1,2},{2,3},{1,4},{4,3},{2,4},{1,5},{6,5},{7,5},{5,4}};
        
        uGraph ug1 = new uGraph(v1,e1,0);
        Farben fb1 = new Farben(ug1);
        fb1.getUgg();
        fb1.fb3();

        //3-färbbar (b)
        int[] v2 = {1,2,3,4,5,6};
        int[][] e2 = {{1,4},{1,5},{1,6},{2,4},{2,5},{2,6},{3,4},{3,5},{3,6}};

        uGraph ug2 = new uGraph(v2,e2,0);
        Farben fb2 = new Farben(ug2);
        fb2.getUgg();
        fb2.fb3();

        //3-färbbar (a), ja
        int[] v3 = {1,2,3,4};
        int[][] e3 = {{1,2},{2,3},{3,4},{4,1}};

        uGraph ug3 = new uGraph(v3,e3,0);
        Farben fb3 = new Farben(ug3);
        fb3.getUgg();
        fb3.bipartit();


        //3-färbbar (a), nein
        int[] v4 = {1,2,3,4,5};
        int[][] e4 = {{1,2},{2,3},{3,4},{4,5},{5,1}};

        uGraph ug4 = new uGraph(v4,e4,0);
        Farben fb4 = new Farben(ug4);
        fb4.getUgg();
        fb4.bipartit();
    }
}
\end{lstlisting}
