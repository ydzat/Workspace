\documentclass[fleqn]{article}

\usepackage{listings}
\usepackage[german]{babel}
\usepackage[T1]{fontenc}
\usepackage[latin1]{inputenc}
\usepackage{titlesec}
\usepackage{geometry}
\usepackage{qtree}
\usepackage{tikz}
\usepackage{amsmath}
\setcounter{secnumdepth}{0}
\usetikzlibrary{positioning}
\geometry{top=2.5cm, bottom=2.5cm}
\lstset{
 columns=fixed,       
 numbers=left,                                        % 在左侧显示行号
 numberstyle=\tiny\color{gray},                       % 设定行号格式
 frame=none,                                          % 不显示背景边框
 backgroundcolor=\color[RGB]{245,245,244},            % 设定背景颜色
 keywordstyle=\color[RGB]{40,40,255},                 % 设定关键字颜色
 numberstyle=\footnotesize\color{darkgray},           
 commentstyle=\it\color[RGB]{0,96,96},                % 设置代码注释的格式
 stringstyle=\rmfamily\slshape\color[RGB]{128,0,0},   % 设置字符串格式
 showstringspaces=false,                              % 不显示字符串中的空格
 language=c++,                                        % 设置语言
 breaklines,                                          % 自动换行
}
\begin{document}

\newpagestyle{main}{
    \sethead{Matrikel-Nr.: 574145 Dongze Yang}{}{Grupe: Mi. Dozent: Julian Pape-Lange}
    \setfoot{}{\thepage}{}
    \headrule
    \footrule
}
\pagestyle{main}
%\section{Aufgabe}

\section{2. Aufgabe}

(a)
$$
%\begin{center}
\begin{tikzpicture}
    \node(1) at (0,0)[circle, draw]{1};
    \node(2) at (1.5,-1.5)[circle, draw]{2};
    \node(3) at (3,-3)[circle, draw]{3};
    \node(1l) at (-1.5,-1.5)[draw]{};
    \node(2l) at (0,-3)[draw]{};
    \node(3l) at (1.5,-4.5)[draw]{};
    \node(3r) at (4.5,-4.5)[draw]{};
    \draw (1)--(2);
    \draw (1)--(1l);
    \draw (2)--(3);
    \draw (2)--(2l);
    \draw (3)--(3l);
    \draw (3)--(3r);
\end{tikzpicture}
\ \ (1) \ \ \ \ \ \ \
\begin{tikzpicture}
    \node(1) at (0,0)[circle, draw]{1};
    \node(2) at (1.5,-1.5)[circle, draw]{3};
    \node(3) at (0,-3)[circle, draw]{2};
    \node(1l) at (-1.5,-1.5)[draw]{};
    \node(2l) at (3,-3)[draw]{};
    \node(3l) at (-1.5,-4.5)[draw]{};
    \node(3r) at (1.5,-4.5)[draw]{};
    \draw (1)--(2);
    \draw (1)--(1l);
    \draw (2)--(3);
    \draw (2)--(2l);
    \draw (3)--(3l);
    \draw (3)--(3r);
\end{tikzpicture}
\ \ (2)
$$
$$
\begin{tikzpicture}
    \node(1) at (0,0)[circle, draw]{2};
    \node(2) at (-1.5,-1.5)[circle, draw]{1};
    \node(3) at (1.5,-1.5)[circle, draw]{3};
    %\node(1l) at (-1.5,-1.5)[draw]{};
    \node(2l) at (-3,-3)[draw]{};
    \node(2r) at (0,-3)[draw]{};
    \node(3l) at (0.3,-3)[draw]{};
    \node(3r) at (3,-3)[draw]{};
    \draw (1)--(2);
    \draw (1)--(3);
    \draw (2)--(2r);
    \draw (2)--(2l);
    \draw (3)--(3l);
    \draw (3)--(3r);
\end{tikzpicture}
\ \ (3) \ \ \ \ \ \ \
\begin{tikzpicture}
    \node(1) at (0,0)[circle, draw]{3};
    \node(2) at (-1.5,-1.5)[circle, draw]{2};
    \node(3) at (-3,-3)[circle, draw]{1};
    \node(1l) at (1.5,-1.5)[draw]{};
    \node(2l) at (0,-3)[draw]{};
    \node(3l) at (-1.5,-4.5)[draw]{};
    \node(3r) at (-4.5,-4.5)[draw]{};
    \draw (1)--(2);
    \draw (1)--(1l);
    \draw (2)--(3);
    \draw (2)--(2l);
    \draw (3)--(3l);
    \draw (3)--(3r);
\end{tikzpicture}
\ \ (4)
%\end{center}
$$
$$
\begin{tikzpicture}
    \node(1) at (0,0)[circle, draw]{3};
    \node(2) at (-1.5,-1.5)[circle, draw]{1};
    \node(3) at (0,-3)[circle, draw]{2};
    \node(1l) at (1.5,-1.5)[draw]{};
    \node(2l) at (-3,-3)[draw]{};
    \node(3l) at (1.5,-4.5)[draw]{};
    \node(3r) at (-1.5,-4.5)[draw]{};
    \draw (1)--(2);
    \draw (1)--(1l);
    \draw (2)--(3);
    \draw (2)--(2l);
    \draw (3)--(3l);
    \draw (3)--(3r);
\end{tikzpicture} \ \ (5) \ \ \ \ 
\begin{tikzpicture}
    \node(1) at (0,0)[circle, draw]{2};
    \node(2) at (-1.5,-1.5)[circle, draw]{3};
    \node(3) at (1.5,-1.5)[circle, draw]{1};
    %\node(1l) at (-1.5,-1.5)[draw]{};
    \node(2l) at (-3,-3)[draw]{};
    \node(2r) at (0,-3)[draw]{};
    \node(3l) at (0.3,-3)[draw]{};
    \node(3r) at (3,-3)[draw]{};
    \draw (1)--(2);
    \draw (1)--(3);
    \draw (2)--(2r);
    \draw (2)--(2l);
    \draw (3)--(3l);
    \draw (3)--(3r);
\end{tikzpicture} \ \ (6)
$$
\\
(b)

(1):  $p_1 =\frac{5}{9}, \ p_2 = \frac{2}{9}, \ p_3 = \frac{2}{9}$.

(2):  $p_1 =\frac{5}{9}, \ p_2 = \frac{2}{9}, \ p_3 = \frac{2}{9}$.

(3):  $p_1 =\frac{4}{10}, \ p_2 = \frac{3}{10}, \ p_3 = \frac{3}{10}$.

(4):  $p_1 =\frac{2}{9}, \ p_2 = \frac{2}{9}, \ p_3 = \frac{2}{9}$.

(5):  $p_1 =\frac{2}{9}, \ p_2 = \frac{2}{9}, \ p_3 = \frac{2}{9}$.

(6):  $p_1 =\frac{3}{10}, \ p_2 = \frac{3}{10}, \ p_3 = \frac{4}{10}$.



\end{document}