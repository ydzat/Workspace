\documentclass[fleqn]{article}

\usepackage{listings}
\usepackage[german]{babel}
\usepackage[T1]{fontenc}
\usepackage[latin1]{inputenc}
\usepackage{titlesec}
\usepackage{geometry}
\usepackage{qtree}
\usepackage{tikz}
\usepackage{amsmath}
\usetikzlibrary{positioning}
\geometry{top=2.5cm, bottom=2.5cm}
\lstset{
 columns=fixed,       
 numbers=left,                                        % 在左侧显示行号
 numberstyle=\tiny\color{gray},                       % 设定行号格式
 frame=none,                                          % 不显示背景边框
 backgroundcolor=\color[RGB]{245,245,244},            % 设定背景颜色
 keywordstyle=\color[RGB]{40,40,255},                 % 设定关键字颜色
 numberstyle=\footnotesize\color{darkgray},           
 commentstyle=\it\color[RGB]{0,96,96},                % 设置代码注释的格式
 stringstyle=\rmfamily\slshape\color[RGB]{128,0,0},   % 设置字符串格式
 showstringspaces=false,                              % 不显示字符串中的空格
 language=c++,                                        % 设置语言
}
\begin{document}

\newpagestyle{main}{
    \sethead{Matrikel-Nr.: 574145 Dongze Yang}{}{Grupe: Mi. Dozent: Julian Pape-Lange}
    \setfoot{}{\thepage}{}
    \headrule
    \footrule
}
\pagestyle{main}
\section{Aufgabe}

\begin{lstlisting}
#include<bits/stdc++.h>
using namespace std;
#define nn 200
vector<int> Map[nn];
int n, vis[nn], match[nn];
bool dfs(int u){
    for(int i = 0; i < Map[u].size(); i++){
        int to = Map[u][i];
        if(!vis[to]){
            vis[to] = 1;
            if(!match[to] || dfs(match[to])){
                match[to] = u;
                return 1;
            }
        }
    }
    return 0;
}
int solve(){
    memset(match, 0, sizeof(match));
    int ans = 0;
    for(int i = 1; i <= n; i++){
        memset(vis, 0, sizeof(vis));
        if(dfs(i)) ans++;
    }
    return ans;
}
void add(int u, int v){
    Map[u].push_back(v);
}
int main(){
    int T, i, m, u, v;
    scanf("%d", &T);
    while(T--){
        scanf("%d %d", &n, &m);
        for(i = 0; i <= n; i++){
            Map[i].clear();
        }
        while(m--){
            scanf("%d %d", &u, &v);
            add(u, v);
        }
        printf("%d\n", n - solve());
    }
    return 0;
}
\end{lstlisting}

\section{Aufgabe}

\begin{tikzpicture}

    \node(s) at (0,3) [circle, draw]{s};
    \node(v1) at (2,4) [circle, draw, label={above:V1}]{v1};
    \node(v2) at (2,3) [circle, draw]{v2};
    \node(v3) at (2,2) [circle, draw]{v3};
    \node(v4) at (2,1) [circle, draw]{v4};
    \node(v5) at (4,4) [circle, draw, label={above:V2}]{v5};
    \node(v6) at (4,3) [circle, draw]{v6};
    \node(v7) at (4,2) [circle, draw]{v7};
    \node(t) at (6,3) [circle, draw]{t};

    \path[->] (s) edge node[above]{/1} (v1)
              (s) edge node[above]{/1} (v2)
              (s) edge node[above]{/1} (v3)
              (s) edge node[below]{/1} (v4)
              (v1) edge[ultra thick] node[above]{} (v5)
              (v1) edge node[above]{} (v6)
              (v2) edge node[above]{} (v7)
              (v3) edge[ultra thick] node[above]{} (v6)
              (v3) edge node[above]{} (v7)
              (v4) edge[ultra thick] node[below]{/$\infty$} (v7)
              (v5) edge node[above]{/1} (t)
              (v6) edge node[above]{/1} (t)
              (v7) edge node[above]{/1} (t)
    ;

\end{tikzpicture}

\section{Aufgabe}

Hier benutze ich KM Algorithmus (genannt auch: Ungarische Methode).

Idee: Wenn ein gleiches Subgraph des Originalgraphen eine vollständige Übereinstimmung enthält, ist diese Übereinstimmung die beste zweigliedrige Übereinstimmung des Originalgraphen.

Beweis: Da die Durchführbarkeit der Vertex-Beschriftung im Algorithmus beibehalten wurde, muss die Summe der Gewichte aller Übereinstimmungen kleiner oder gleich der Summe der möglicher Vertex-Beschriftung aller Knoten sein, und die vollständige Übereinstimmung im gleichen Teilgraphen muss die beste Übereinstimmung sein.

Nehmen Sie den Graphe in Aufgabe 2 als Beispiel:

Weil jeder Person bis zu zwei Tätigkeiten gleichzeitig zugemutet werden können, sind die Vertex-Beschriftungen von $v_1, v_2, v_3, v_4$ 2, sind die die Vertex-Beschriftungen von $v_5, v_6, v_7$ 1.

dann gilt: $weight(v_1,v_5) = weight(v_1,v_6) = weight(v_2,v_7) = weight(v_3,v_6) = weight(v_3,v_7) = weight(v_4,v_7) = 1$.
Das aktuelle Match befriedigt $lv1(u) + lv2(v) \geq weight(u,v)$, wobei $lv1(u)$ besitzt sich auf der Vertex-Beschriftung der Knote in der Menge $V_1$, $lv2(v)$-> $V_2$.

so: $\sum_{i=1}^{u_i \in V1}lv1(u_i) + \sum_{i=1}^{v_i \in V2}lv2(v_i) = K \geq \sum weright(u_i,v_i)$


\end{document}