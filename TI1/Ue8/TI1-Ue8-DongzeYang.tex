\documentclass[fleqn]{article}

\usepackage[german]{babel}
\usepackage[T1]{fontenc}
\usepackage[latin1]{inputenc}
\usepackage{titlesec}
\usepackage{geometry}
\usepackage{qtree}
\usepackage{tikz}
\usepackage{amsmath}
\usetikzlibrary{positioning}
\geometry{top=2.5cm, bottom=2.5cm}
\begin{document}

\newpagestyle{main}{
    \sethead{Matrikel-Nr.: 574145 Dongze Yang}{}{Grupe: Mi. Dozent: Julian Pape-Lange}
    \setfoot{}{\thepage}{}
    \headrule
    \footrule
}
\pagestyle{main}
\section{Aufgabe}


Begin:
$$A_{-1}=\begin{bmatrix}
    0 & 4 & 2 & \infty \\
    \infty & 0 & 5 & -3 \\
    \infty & \infty & 0 & 1 \\
    \infty & \infty & \infty & 0
\end{bmatrix} \ \ \ \ \
Path_{-1}=\begin{bmatrix}
    -1 & -1 & -1 & -1 \\
    -1 & -1 & -1 & -1 \\
    -1 & -1 & -1 & -1 \\
    -1 & -1 & -1 & -1
\end{bmatrix}
$$  

(1,1),(1,2),(1,3),(1,4),(2,1),(2,3),(2,4),(3,1),(3,2),(3,4),(4,1),(4,2),(4,3)  

if( \ A[i][k] \ + \ A[k][j] \ < \ A[i][j] \ ) $\{$

\qquad A[i][j] \ = \ A[i][k] \ + \ A[k][j];

$\}$
else continue;  

wobei zeigen i und j auf zwei Elemente im Tupel, z.B für (0,1), i $\rightarrow$ 0, j $\rightarrow$ 1. 
k zeigt auf einen Zwischenknoten.  

Laß k = 1. Gibt es keine \ A[i][k] \ + \ A[k][j] \ < \ A[i][j] \ . Deswegen sieht $A_{1}$ und $Path_{1}$ so aus:

$$A_{1}=\begin{bmatrix}
    0 & 4 & 2 & \infty \\
    \infty & 0 & 5 & -3 \\
    \infty & \infty & 0 & 1 \\
    \infty & \infty & \infty & 0
\end{bmatrix} \ \ \ \ \
Path_{1}=\begin{bmatrix}
    -1 & -1 & -1 & -1 \\
    -1 & -1 & -1 & -1 \\
    -1 & -1 & -1 & -1 \\
    -1 & -1 & -1 & -1
\end{bmatrix}
$$
k = 1 ist nicht als Zwischenknoten geeignet.  
\\  

Laß k = 2.

Gibt es A[1][2] \ + \ A[2][4] \ < \ A[1][4].
Deswegen sieht $A_{2}$ und $Path_{2}$ so aus:

$$A_{2}=\begin{bmatrix}
    0 & 4 & 2 & -1 \\
    \infty & 0 & 5 & -3 \\
    \infty & \infty & 0 & 1 \\
    \infty & \infty & \infty & 0
\end{bmatrix} \ \ \ \ \
Path_{2}=\begin{bmatrix}
    -1 & -1 & -1 & 2 \\
    -1 & -1 & -1 & -1 \\
    -1 & -1 & -1 & -1 \\
    -1 & -1 & -1 & -1
\end{bmatrix}
$$
\\  

Laß k = 3.

Gibt es keine A[i][3] \ + \ A[3][j] \ < \ A[i][j].
Deswegen sieht $A_{3} = A_{2}$ und $Path_{3} = Path_{2}$  so aus:

$$A_{3}=\begin{bmatrix}
    0 & 4 & 2 & -1 \\
    \infty & 0 & 5 & -3 \\
    \infty & \infty & 0 & 1 \\
    \infty & \infty & \infty & 0
\end{bmatrix} \ \ \ \ \
Path_{3}=\begin{bmatrix}
    -1 & -1 & -1 & 2 \\
    -1 & -1 & -1 & -1 \\
    -1 & -1 & -1 & -1 \\
    -1 & -1 & -1 & -1
\end{bmatrix}
$$
\\

Laß k = 4.

Gibt es keine A[i][4] \ + \ A[4][j] \ < \ A[i][j].
Deswegen sieht $A_{4} = A_{3}$ und $Path_{4} = Path_{3}$  so aus:

$$A_{4}=\begin{bmatrix}
    0 & 4 & 2 & -1 \\
    \infty & 0 & 5 & -3 \\
    \infty & \infty & 0 & 1 \\
    \infty & \infty & \infty & 0
\end{bmatrix} \ \ \ \ \
Path_{4}=\begin{bmatrix}
    -1 & -1 & -1 & 2 \\
    -1 & -1 & -1 & -1 \\
    -1 & -1 & -1 & -1 \\
    -1 & -1 & -1 & -1
\end{bmatrix}
$$

\section{Aufgabe}
(a)  

Begin:
$$A_{-1}=\begin{bmatrix}
    0 & 1 & \infty & \infty \\
    \infty & 0 & 1 & 1 \\
    \infty & -1 & 0 & \infty \\
    \infty & \infty & \infty & 0
\end{bmatrix} \ \ \ \ \
Path_{-1}=\begin{bmatrix}
    -1 & -1 & -1 & -1 \\
    -1 & -1 & -1 & -1 \\
    -1 & -1 & -1 & -1 \\
    -1 & -1 & -1 & -1
\end{bmatrix}
$$
\\

k = 1:
$$A_{1}=\begin{bmatrix}
    0 & 1 & \infty & \infty \\
    \infty & 0 & 1 & 1 \\
    \infty & -1 & 0 & \infty \\
    \infty & \infty & \infty & 0
\end{bmatrix} \ \ \ \ \
Path_{1}=\begin{bmatrix}
    -1 & -1 & -1 & -1 \\
    -1 & -1 & -1 & -1 \\
    -1 & -1 & -1 & -1 \\
    -1 & -1 & -1 & -1
\end{bmatrix}
$$
\\

k = 2:

Weil es eine negative Gewichtsschleife (2 $\rightarrow$ 3 $\rightarrow$ 2) gibt, bei jeder Lauf entlang dieses Rings wird der kürzeste Pfad(a $\rightarrow$ ... $\rightarrow$ c und b $\rightarrow$ ... $\rightarrow$ c und c $\rightarrow$ ... $\rightarrow$ b) um 1 verringert, dann der kürzeste Pfad wird nie gefunden.

Der Algorithmus sollte zu diesem Zeitpunkt beendet werden.


\end{document}