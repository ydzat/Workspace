\documentclass[fleqn]{article}
\usepackage[UTF8]{ctex}
\usepackage{listings}
\usepackage{diagbox}
\usepackage[german]{babel}
\usepackage[T1]{fontenc}
\usepackage[latin1]{inputenc}
\usepackage{titlesec}
\usepackage{geometry}
\usepackage{qtree}
\usepackage{fontspec}
\setmonofont{Consolas}
\usepackage{tikz}
\usepackage{amsmath}
\usepackage{amssymb}
\setcounter{secnumdepth}{0}
\usetikzlibrary{positioning}
\geometry{top=2.5cm, bottom=2.5cm}
\lstset{
 columns=fixed,       
 numbers=left,                                        % 在左侧显示行号
 numberstyle=\tiny\color{gray},                       % 设定行号格式
 frame=none,                                          % 不显示背景边框
 %backgroundcolor=\color[RGB]{245,245,244},            % 设定背景颜色
 keywordstyle=\color[RGB]{40,40,255},                 % 设定关键字颜色
 numberstyle=\footnotesize\color{darkgray},           
 commentstyle=\it\color[RGB]{0,96,96},                % 设置代码注释的格式
 stringstyle=\rmfamily\slshape\color[RGB]{128,0,0},   % 设置字符串格式
 showstringspaces=false,                              % 不显示字符串中的空格
 language=c++,                                        % 设置语言
 breaklines,                                          % 自动换行
}
\begin{document}

%\maketitle



\newpagestyle{main}{
    \sethead{}{}{Rechnernetze}
    \setfoot{}{\thepage}{}
    \headrule
    \footrule
}
\pagestyle{main}
\section{邮件客户端(09)}

\begin{lstlisting}[language = Python, numbers=left, 
    numberstyle=\tiny,keywordstyle=\color{blue!70},
    commentstyle=\color{red!50!green!50!blue!50},frame=shadowbox,
    rulesepcolor=\color{red!20!green!20!blue!20},basicstyle=\ttfamily]
import poplib

HOST = 'taurus.informatik.tu-chemnitz.de'
PORT = 110

username = "rot"
password = "rot"

client = poplib.POP3(HOST, PORT)

usr = client.user(username) #legt den gewünschten Nutzernamen fest
print(usr.decode())

pas = client.pass_(username) #authentifiziert den Nutzer
print(pas.decode())

ret = client.list() #gibt als Liste die Nummern und Größen der verfügbaren EMails zurück

print(ret[0].decode())

text = client.retr(1) #liefert die EMail mit der angegebenen Nummer zurück

for i in text[1]:
    print(i.decode())

client.quit() #beendet die Verbindung
\end{lstlisting}




\clearpage

\section{服务器验证ip地址(18)}
\noindent 服务端
\begin{lstlisting}[language = Python, numbers=left, 
    numberstyle=\tiny,keywordstyle=\color{blue!70},
    commentstyle=\color{red!50!green!50!blue!50},frame=shadowbox,
    rulesepcolor=\color{red!20!green!20!blue!20},basicstyle=\ttfamily]
server = socket.socket(socket.AF_INET, socket.SOCK_STREAM)
HOST = 'localhost'
Port = 26718
#计算主机134.109.133.128/26所属网络
#26-8-8-8=2 -> 11111111.11111111.11111111.11000000 -> 255.255.255.192
#范围为128-191
valid_low = ipaddress.ip_address("134.109.133.128")
valid_high = ipaddress.ip_address("134.109.133.191")
server.bind((HOST, 26718))
server.listen(5)
while True:
    conn, addr = server.accept()
    usr = ipaddress.ip_address(addr[0])
    if ( valid_low <= usr <= valid_high):
        msg = 'Access granted' 
    else:
        msg = 'Access denied'
    conn.send(msg.encode())
    conn.close()
server.close()
\end{lstlisting}

\noindent 客户端
\begin{lstlisting}[language = Python, numbers=left, 
    numberstyle=\tiny,keywordstyle=\color{blue!70},
    commentstyle=\color{red!50!green!50!blue!50},frame=shadowbox,
    rulesepcolor=\color{red!20!green!20!blue!20},basicstyle=\ttfamily]
import socket
client = socket.socket(socket.AF_INET, socket.SOCK_STREAM)
HOST = '134.109.133.128'
client.connect((HOST, 26718))
data = client.recv(1024)
print(data.decode())
client.close()
\end{lstlisting}

\clearpage
\section{服务器响应(17)}
\noindent 通过浏览器访问指定服务器,若存在页面则返回ip地址,若不存在则返回404

\begin{lstlisting}[language = Python, numbers=left, 
    numberstyle=\tiny,keywordstyle=\color{blue!70},
    commentstyle=\color{red!50!green!50!blue!50},frame=shadowbox,
    rulesepcolor=\color{red!20!green!20!blue!20},basicstyle=\ttfamily]
import socket
HOST = "25.150.140.133"
Port = 8080
server = socket.socket(socket.AF_INET, socket.SOCK_STREAM)
server.bind((HOST, Port))
server.listen(1)
conn, addr = server.accept()
#print("Verbund mit: ", addr[0])
web = ("""HTTP/1.1 200 OK \n\n<!doctype html>
<html>
    <head><title>Hello</title></head>
    <body>
        <H2>Hello</H2>
        your IP address is: """+addr[0]+"""
    </body>
</html>
""" )

errweb = ("""HTTP/1.1 200 OK \n\n <!doctype html>
<html>
    <head><title>404</title></head>
    <body>Site not found on this server!</body>
</html>
""" )

while 1:
    data = conn.recv(1024)
    datasplit = data.decode().split()
    if ((datasplit[0] == "GET") and (datasplit[1] == "/" or datasplit[1] == "/index.html") and (datasplit[2] == "HTTP/1.1")):
        conn.send(web.encode())
        break
    else:
        conn.send(errweb.encode())
        break
conn.close()
server.close()
    
\end{lstlisting}

\section{HTML+PY脚本(14)}
\noindent 用py作服务器,处理来自HTML的数据,并更新HTML

\noindent 脚本部分
\begin{lstlisting}[language = Python, numbers=left, 
    numberstyle=\tiny,keywordstyle=\color{blue!70},
    commentstyle=\color{red!50!green!50!blue!50},frame=shadowbox,
    rulesepcolor=\color{red!20!green!20!blue!20},basicstyle=\ttfamily]
import socket
import re

path="C:\\Users\\ydzat\\OneDrive\\Workspace\\Rechnernetze\\code\\ans2014\\new\\index.html"

f = open(path,'r+')
s = socket.socket(socket.AF_INET, socket.SOCK_STREAM)

host = '25.150.140.133'
port = 8080

s.bind((host,port))
s.listen(1)

stock = 0

for line in f.readlines():
    da = re.search(r'Still (\d+)', line, re.M)
    if (type(da) == re.Match ):
        stock = da.group(1)
        break
    
while True:
    data = ''
    conn, addr = s.accept()
    msg = conn.recv(1024)
    userRequest = msg.decode().split()
    if userRequest[1] == "/one":
        stock = int(stock) - 1
    elif userRequest[1] == "/five":
        stock = int(stock) - 5
    elif userRequest[1] == "/ten":
        stock = int(stock) - 10
    f.seek(0, 0)
    for line in f.readlines():
        if (type(re.search(r'Still (\d+)', line, re.M))==re.Match):
            line = 'Still ' + str(stock) + " pieces\n" 
        data += line
    data = "\"\"\"HTTP/1.1200OK\n\n"+ data
    #print(data)
    #f.seek(0,0)
    #f.writelines(data)
    conn.send(data.encode())
    conn.close()
    
conn.close()
s.close()
\end{lstlisting}

\noindent HTML部分

\begin{lstlisting}[language = Python, numbers=left, 
    numberstyle=\tiny,keywordstyle=\color{blue!70},
    commentstyle=\color{red!50!green!50!blue!50},frame=shadowbox,
    rulesepcolor=\color{red!20!green!20!blue!20},basicstyle=\ttfamily]
<html>
<head>
    <link rel="stylesheet" type="text/css" href="index.css">
    <meta charset="UTF-8">
</head>
<body>
    <table class="abc">
        <tr><td>Shop<HR></td></tr>
        <tr><td action="http://25.150.140.133:8080">
                <a href="http://25.150.140.133:8080/one">One piece</a><br/>
                <a href="http://25.150.140.133:8080/five">Five piece</a><br/>
                <a href="http://25.150.140.133:8080/ten">Ten piece</a>
    </td></tr></table>
    <table class="abc">
        <tr><td>Remaining<hr></td></tr>
        <tr><td>
                Still 1000 pieces
    </td></tr></table>
</body></html>
\end{lstlisting}

\noindent CSS部分
\begin{lstlisting}[language = Python, numbers=left, 
    numberstyle=\tiny,keywordstyle=\color{blue!70},
    commentstyle=\color{red!50!green!50!blue!50},frame=shadowbox,
    rulesepcolor=\color{red!20!green!20!blue!20},basicstyle=\ttfamily]
table.abc{
    border : 2px solid black;
    border-style:solid;
    padding:0;
    margin:10;
}
\end{lstlisting}

\section{服务器记录客户端发送某数据的次数(15)}
\noindent 服务端
\begin{lstlisting}[language = Python, numbers=left, 
    numberstyle=\tiny,keywordstyle=\color{blue!70},
    commentstyle=\color{red!50!green!50!blue!50},frame=shadowbox,
    rulesepcolor=\color{red!20!green!20!blue!20},basicstyle=\ttfamily]
import socket

def getNum(msg):
    if (msg in data.keys()):
        data[msg] = data[msg] + 1
        print(data[msg])
    else:
        data[msg] = 1
    return data[msg] 

server = socket.socket(socket.AF_INET, socket.SOCK_STREAM)

host = 'localhost'
port = 8080

server.bind((host,port))

server.listen(1)
data = {}
conn, addr = server.accept()
while 1:
    msg = getNum(conn.recv(1024).decode())
    conn.send(str(msg).encode())
conn.close()
server.close()
\end{lstlisting}

\noindent 客户端
\begin{lstlisting}[language = Python, numbers=left, 
    numberstyle=\tiny,keywordstyle=\color{blue!70},
    commentstyle=\color{red!50!green!50!blue!50},frame=shadowbox,
    rulesepcolor=\color{red!20!green!20!blue!20},basicstyle=\ttfamily]
import socket
client = socket.socket(socket.AF_INET, socket.SOCK_STREAM)
host = "localhost"
port = 8080
client.connect((host, port))
while 1:
    msg = input()
    client.send(msg.encode())
    count = client.recv(1024)
    print(count.decode())
client.close()
\end{lstlisting}

\clearpage
\section{服务器计算(12)}
\noindent 客户通过浏览器访问服务器,通过直接输入指定地址作为操作符合数字。浏览器计算并返回结果。

$http: //< Rechnername >:< port >/< operation >/< operand1>/< operand2>$

\begin{lstlisting}[language = Python, numbers=left, 
    numberstyle=\tiny,keywordstyle=\color{blue!70},
    commentstyle=\color{red!50!green!50!blue!50},frame=shadowbox,
    rulesepcolor=\color{red!20!green!20!blue!20},basicstyle=\ttfamily]
import socket
PORT=8080
s=socket.socket(socket.AF_INET,socket.SOCK_STREAM)
s.bind(("localhost",PORT))
s.listen(1)
data=[]
count=[]
def count_num(from_c):
    index = 0
    for i in range(len(data)):
        if data[i]==from_c:
            index = i
    if from_c not in data:
        data.append(from_c)
        count.append(1)
        return "1"
    else:
        count[index] = count[index] + 1
        return str(count[index])
while True:
    conn,addr = s.accept()
    msg = conn.recv(1024)
    sendmsg = count_num(msg.decode())
    conn.send(sendmsg.encode())
    conn.close()
s.close()
\end{lstlisting}

\clearpage
\section{交换机Switch(05)}

Der "Switch" besteht aus N Anschlüssen. “交换机”由N个连接组成。

Zur Vereinfachung gibt es maximal 255 Adressen, wobei die Adresse 255 als Broadcast-Adresse verwendet wird.
为简化起见,最多有255个地址,地址255用作广播地址。

Es wird ein Paketkopf gelesen und entsprechend des Zustandes der Engine entschieden, auf welchen Anschluss das zugehörige Paket geschickt wird.
读取包头,并确定引擎的哪个状态到相应数据包的端口。

Informationen, die an die Engine geliefert werden, sind Eingangsportnummer (1..N), Absenderadresse (1..255) und Zieladresse (1..255).
提供给引擎的信息是输入端口号(1..N),发送方地址(1..255)和目标地址(1..255)。

Die Engine bestimmt daraufhin das Ziel und gibt diese Information aus.
然后引擎确定目的地并输出此信息。

Mit der Eingabe von "\,a" wird die Ausgabe der Adresstabellen ausgelöst.
输入“a”会触发地址表的输出。

6端口交换机的可能示例会话:
\begin{center}
\fbox{
    \shortstack[l]{
switch< 1 23 54\\
switch> Ausgabe auf allen Ports\\
switch< 4 32 23 \\
switch> Ausgabe auf Port 1 \\
switch< 6 35 32 \\
switch> Ausgabe auf Port 4 \\ 
switch< 2 32 23 \\
switch> Ausgabe auf Port 1 \\
switch< 6 85 32 \\
switch> Ausgabe auf Port 2\\
switch< 1 55 35 \\
switch> Ausgabe auf Port 6 \\
switch< 6 5 255 \\
switch> Ausgabe auf allen Ports \\
switch< a \\
switch> 
    }
}
\fbox{\shortstack[l]{1: 23 55 \\
2: 32 \\
3: \\
4: \\
5: \\
6: 35 85 5}}    
\end{center}

\begin{lstlisting}[language = Python, numbers=left, 
    numberstyle=\tiny,keywordstyle=\color{blue!70},
    commentstyle=\color{red!50!green!50!blue!50},frame=shadowbox,
    rulesepcolor=\color{red!20!green!20!blue!20},basicstyle=\ttfamily]
data = [int(x) for x in input('switch<').split()]
port =[[] for i in range(6)]
print(port)
def find(a,b):
    n = len(a)
    for i in range(0,n):
        m = len(a[i])
        for j in range(0,m):
            if b == a[i][j]:
                return i+1
    return -1
def aging(a,b,c):
    n = len(a)
    for i in range(0,n):
        m = len(a[i])
        for j in range(0,m):
            if b == a[i][j] and c != i:
                a[i].pop(j)
                return
while data[0] != 'a':
    print(data)
    (port[data[0]-1]).append(data[1])
    num = find(port,data[2])
    aging(port,data[1],data[0]-1)
    if num == -1:
        print("switch>Ausgabe auf allen Ports")
    else:
        print("switch>Ausgabe auf Port",end=" ")
        print(num)
    print(port)
    data = [int(x) for x in input('switch<').split()]
print("switch>")
n = len(port)
for i in range(0,n):
    print("i+1:",end=" ")
    m = len(port[i])
    for j in range(0,m):
        print(port[i][j],end="")
    print()
\end{lstlisting}

\clearpage
\section{UDP}
\noindent 面向非连接的,不可靠的,少量数据,快。主流通过UDP发送

\begin{center}
    Server
\end{center}
\begin{lstlisting}[language = Python, numbers=left, 
    numberstyle=\tiny,keywordstyle=\color{blue!70},
    commentstyle=\color{red!50!green!50!blue!50},frame=shadowbox,
    rulesepcolor=\color{red!20!green!20!blue!20},basicstyle=\ttfamily]
import sys,socket

HOST = "localhost"
PORT = 8080
try:
    s = socket.socket(socket.AF_INET,socket.SOCK_DGRAM)
    s.bind((HOST, PORT))
    msg,adr = s.recvfrom(1000)
    print "Empfange von Host %s, port %d: %s"%(adr[0], adr[1], msg)
    s.sendto("+++ " + msg + " +++", adr)
except:
    print "Netzwerkfehler"
    sys.exit(1)
\end{lstlisting}

\begin{center}
    Client
\end{center}
\begin{lstlisting}[language = Python, numbers=left, 
    numberstyle=\tiny,keywordstyle=\color{blue!70},
    commentstyle=\color{red!50!green!50!blue!50},frame=shadowbox,
    rulesepcolor=\color{red!20!green!20!blue!20},basicstyle=\ttfamily]
import sys,socket

if len(sys.argv) != 3:
        print "Benutzung %s  <port>" % sys.argv[0]
        sys.exit(1)
prog,host,port = sys.argv
try:
    n_port = int(port)
except:
    print "%s ist kein gueltiget Port"%port
    sys.exit(1)
try:
    s = socket.socket(socket.AF_INET,socket.SOCK_DGRAM)
    s.sendto("Hallo hallo", (host, n_port))
    msg,adr = s.recvfrom(1000)
    print "Empfange von Host %s, port %d: %s"%(adr[0], adr[1], msg)
except:
    print "Netzwerkfehler"
\end{lstlisting}


\clearpage
\section{TCP(常用版)}
\noindent 面向连接的,可靠,大量数据,慢。发送重要信息

TCP要求在目标计算机成功收到数据时发回一个确认(即 ACK)。如果在某个时限内未收到相应的 ACK,将重新传送数据包。如果网络拥塞,这种重新传送将导致发送的数据包重复。但是,接收计算机可使用数据包的序号来确定它是否为重复数据包,并在必要时丢弃它。

\begin{center}
    Server
\end{center}
\begin{lstlisting}[language = Python, numbers=left, 
    numberstyle=\tiny,keywordstyle=\color{blue!70},
    commentstyle=\color{red!50!green!50!blue!50},frame=shadowbox,
    rulesepcolor=\color{red!20!green!20!blue!20},basicstyle=\ttfamily]
import socket,sys

if len(sys.argv) != 2:
    print ("Benutzung: python %s <port>"%sys.argv[0])
    sys.exit(1)
try:
    port = int(sys.argv[1])
except:
    print ("%s ist keine gueltige Portnummer"%sys.argv[1])
    sys.exit(1)
try:
    s = socket.socket(socket.AF_INET, socket.SOCK_STREAM)
    s.bind(("", port))
    s.listen(1)
    while 1:
    conn, addr = s.accept()
    print "Verbindung von Host: %s, port %d"%(addr[0], addr[1])
    data = conn.recv(1024)
    if not data: break
    print "empfangen: %s"%data
    conn.send("+++ " + data + " +++")
    conn.close()
    s.close()
except:
    print "Server error"
    sys.exit(1)
\end{lstlisting}

\begin{center}
    Client
\end{center}
\begin{lstlisting}[language = Python, numbers=left, 
    numberstyle=\tiny,keywordstyle=\color{blue!70},
    commentstyle=\color{red!50!green!50!blue!50},frame=shadowbox,
    rulesepcolor=\color{red!20!green!20!blue!20},basicstyle=\ttfamily]
import socket, sys

if len(sys.argv) != 3:
    print "Benutzung python %s <port>"%sys.argv[0]
    sys.exit(1)
try:
    port = int(sys.argv[2])
except:
    print "%s ist keine gueltige Portnummer"%sys.argv[2]
    sys.exit(1)
try:
    s = socket.socket(socket.AF_INET, socket.SOCK_STREAM)
    s.connect((sys.argv[1], port))
    s.send('Hello, world')
    data = s.recv(1024)
    print "empfangen: %s"%data
    s.close()
except:
    print "Verbindung abgelehnt"
    sys.exit(1)
\end{lstlisting}

\clearpage
\section{网页邮箱客户端}
\begin{lstlisting}[language = Python, numbers=left, 
    numberstyle=\tiny,keywordstyle=\color{blue!70},
    commentstyle=\color{red!50!green!50!blue!50},frame=shadowbox,
    rulesepcolor=\color{red!20!green!20!blue!20},basicstyle=\ttfamily]
import poplib
import socket
import re

def akt(client, web):
    
    lists = client.list()
    
    numOfmail = int(lists[0].decode().split()[1])
    data = ""
    
    for i in range(numOfmail,0,-1):

        resp, lines, octets = client.top(i,0)
        msg_content = b'\r\n'.join(lines).decode('utf-8')
        dateT = re.search(r'(^Date:)(.+)', str(msg_content),re.M)
        date.append(dateT.group(2).rstrip())
        betreffT = re.search(r'(^Subject:)(.+)', str(msg_content), re.M)
        betreff.append(betreffT.group(2).rstrip())
        senderT = re.search(r'(^From:)(.+)', str(msg_content), re.M)
        senderT = senderT.group(2).rstrip().replace('<','&lt;').replace('>','&gt;')
        sender.append(senderT)

    i = 0
    flag = 2
    for line in web.splitlines():
        i += 1
        data = data + line + '\n'
        if (i == 1):
            data = data + '\n\n'
        if (type(re.search("Sender", line)) == re.Match):
            data = data + "</tr>"
            for j in range(numOfmail):
                data = data + "<tr><td><a>" + date[j] + "</a></td><td><a href="">" + betreff[j] + "</a></td><td><a>" + sender[j] + "</a></td></tr><br/>\n"
                
            data = data + "</table></body></html>"
            break
    return data

date = []
betreff = []
sender = []

HOST = 'taurus.informatik.tu-chemnitz.de'
PORT = 110

sHost = 'localhost'
sPort = 8080

web = """HTTP/1.1 200 OK\n\n
<html>
    <head><h1>Webmailer Uebersicht</h1></head>
    <body>
        <table>
            <tr>
                <td action="http://localhost:8080">
                    <a href="http://localhost:8080/aktuallisieren">Aktuallisieren</a></br>
                    <a href="http://localhost:8080/beenden">Server beenden</a></br>
                </td>
            </tr>
            <tr>
                <td>Datum</td>
                <td>Betreff</td>
                <td>Sender</td>
            </tr>
            <tr>
                None
            </tr>
        </table>
    </body>
</html>
"""

client = poplib.POP3(HOST, PORT)
server = socket.socket(socket.AF_INET,socket.SOCK_STREAM)
server.bind((sHost, sPort))
server.listen(1)
client.user("rot\r\n")
client.pass_("rot\r\n")
conn, addr = server.accept()
web = akt(client,web)
conn.send(web.encode())
conn.close()

while 1:
    conn, addr = server.accept()
    msg = conn.recv(1024).decode().split()
    if (msg[1] == "/aktuallisieren"):
        web = akt(client, web)
    if(msg[1]=="beenden"):
        client.quit()
        conn.close()
        server.close()
        break
    conn.send(web.encode())
    conn.close()
client.quit()  #beendet die Verbindung
server.close()
\end{lstlisting}
% \begin{lstlisting}[language = Python, numbers=left, 
%     numberstyle=\tiny,keywordstyle=\color{blue!70},
%     commentstyle=\color{red!50!green!50!blue!50},frame=shadowbox,
%     rulesepcolor=\color{red!20!green!20!blue!20},basicstyle=\ttfamily]

% \end{lstlisting}






















































\end{document}